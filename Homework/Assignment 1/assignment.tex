\documentclass{article}

\usepackage[sans, stdmargin, noindent]{../../rajeev}
\usepackage{mathtools}
\usepackage{dirtytalk}

\pagestyle{fancy}
\rhead{\today}
\lhead{Math 291H HW \#1}

\begin{document}

\begin{center}
    \Large \textbf{Math 291H Homework \#1}
\end{center}
\begin{center}
    \Large Rajeev Atla
\end{center}


Honors Pledge Statement: \say{The writeup of this submission is my own work alone.}

\problem{1.1}

\begin{align*}
  \pars{3, -1} &= s \pars{2, 1} + t \pars{1, 3} \\
  \pars{3, -1} &= \pars{2s+t, s+3t} \\
\end{align*}

We turn this into a system of equations.
$$
\begin{cases}
  2s + t = 3 \\
  s+3t = -1 \\
\end{cases}
$$

Solving, we find that \boxed{s=2, t=-1}.


\problem{1.4}

$$\norm{\bm{x}} = \sqrt{4^2 + 7^2 + \pars{-4}^2 + 1^2 + 2^2 + \pars{-2}^2} =  \boxed{\sqrt{15}}$$

$$\norm{\bm{y}} = \sqrt{2^2 + 1^2 + 2^2 + 2^2 + \pars{-1}^2 + \pars{-1}^2} = \boxed{\sqrt{11}}$$

\begin{align*}
  \cos \theta &= \frac{\bm{x} \cdot \bm{y}}{\norm{\bm{x}} \norm{\bm{y}}} \\
              &= \frac{4 \cdot 2 + 7 \cdot 1 + \pars{-4} \cdot 2 + 1 \cdot \pars{-1} + \pars{-2} \cdot \pars{-1}}{\sqrt{15} \cdot \sqrt{11}} \\
              &= \frac{8}{\sqrt{165}} \\
              &= \frac{8 \sqrt{165}}{165} \\
  \theta &= \boxed{\arccos{\frac{8 \sqrt{165}}{165}}} \\
\end{align*}


\problem{1.5}

$$\norm{\bm{x}} = \sqrt{4^2 + 7^2 + 4^2} =  \boxed{9}$$

$$\norm{\bm{y}} = \sqrt{2^2 + 1^2 + 2^2} =  \boxed{3}$$

\begin{align*}
  \cos \theta &= \frac{4 \cdot 2 + 7 \cdot 1 + 4 \cdot 2}{9 \cdot 3} \\
              &= \frac{23}{27} \\
  \theta &= \boxed{\arccos \frac{23}{27}} \\
\end{align*}



\problem{1.7}

\subsection*{a}

\begin{align*}
  \bm{u}_1 \cdot \bm{u}_2 &= \frac{1}{81} \pars{1 \cdot 8 + \pars{-4} \cdot 4 + \pars{-8} \cdot \pars{-1}} = 0 \\
  \bm{u}_2 \cdot \bm{u}_3 &= \frac{1}{81} \pars{8 \cdot 4 + 4 \cdot \pars{-7} + \pars{-1} \cdot 4} = 0 \\
  \bm{u}_1 \cdot \bm{u}_3 &= \frac{1}{81} \pars{1 \cdot 4 + \pars{-4} \cdot \pars{-7} + \pars{-8} \cdot 4} = 0 \\
\end{align*}

Therefore, $\set{\bm{u}_1, \bm{u}_2, \bm{u}_3}$ \emph{is} an orthonormal basis of $\RR^3$.

\begin{align*}
  \bm{u}_1 \times \bm{u}_2 &= \frac{1}{81} \pars{1, -4, -8} \times \pars{8, 4, -1} \\
                           &= \frac{1}{81} \pars{36, -63, 36} \\
                           &= \frac{1}{9} \pars{4, -7, 4} \\
                           &= \bm{u}_3 \\
\end{align*}
Since $\bm{u}_1 \times \bm{u}_2 = \bm{u}_3$, $\set{\bm{u}_1, \bm{u}_2, \bm{u}_3}$ is a \emph{right-handed} orthonormal basis of $\RR^3$.


\subsection*{b}


\begin{align*}
  y_1 \bm{u}_1 + y_2 \bm{u}_2 + y_3 \bm{u}_3 &= \pars{10, 11, -11} \\
  y_1 \pars{1, -4, -8} + y_2 \pars{8, 4, -1} + y_3 \pars{4, -7, 4} &= \pars{90, 99, -99} \\
  \pars{y_1 + 8 y_2 + 4 y_3, -4 y_1 + 4 y_2 - 7 y_3, -8 y_1 - y_2 + 4y_3} &= \pars{90, 99, -99} \\
\end{align*}

We can solve this system to find \boxed{y_1 = 6, y_2 = 15, y_3 = -9}.

\begin{align*}
  \norm{\pars{y_1, y_2, y_3}} &= \sqrt{6^2 + 15^2 + \pars{-9}^2} \\
                              &= \boxed{3 \sqrt{38}} \\
  \norm{\pars{10, 11, -11}} &= \sqrt{10^2 + 11^2 + \pars{-11}^2} \\
  &= \boxed{3 \sqrt{38}} \\
\end{align*}



\problem{1.14}


\subsection*{a}

Let $\bm{v}_1$ be the vector that passes through $\bm{a}_1$ and $\bm{a}_2$.
Let $\bm{v}_2$ be the vector that passes through $\bm{a}_2$ and $\bm{a}_3$.
Let $\bm{v}_3$ be the vector that passes through $\bm{b}_1$ and $\bm{b}_2$.
Let $\bm{v}_4$ be the vector that passes through $\bm{b}_2$ and $\bm{b}_3$.

\begin{align*}
  \bm{v}_1 &= \pars{-2, 0, -4} \\
  \bm{v}_2 &= \pars{3, -5, 1} \\
  \bm{v}_3 &= \pars{0, -1, 1} \\
  \bm{v}_4 &= \pars{-1, 1, 0} \\
\end{align*}

Since $\bm{v}_1, \bm{v}_2$ lie in the plane $P_1$, their cross product $\bm{n}_1$ is perpendicular to $P_1$.
Likewise for $\bm{v}_3, \bm{v}_4, P_2, \text{and }\bm{n}_2$, respectively.

\begin{align*}
  \bm{n}_1 &= \pars{20, -10, 10} \\
  \bm{n}_2 &= \pars{-1, -1, -1} \\
\end{align*}

We substitute into the standard form equation for a plane:

\begin{align*}
  P_1 : \bm{n}_1 \cdot \bm{r} + d_1 &= 0 \\
  P_2 : \bm{n}_2 \cdot \bm{r} + d_2 &= 0 \\
\end{align*}

Substituting $\bm{a}_1$ and $\bm{b}_1$, respectively, we find that $d_1 = -10$ and $d_2 = 3$.
After simplifying,

\begin{align*}
  P_1 &: \boxed{2x - y + z - 1 = 0} \\
  P_2 &: \boxed{x + y + z - 3 = 0} \\
\end{align*}


\subsection*{b}

Adding and subtracting the two equations, respectively

$$
\begin{cases}
  x - 2y + 2 = 0 \Longleftrightarrow y = \frac{1}{2} x + 1 \\
  3x + 2z - 4 = 0 \Longleftrightarrow z = - \frac{3}{2} x + 2 \\
\end{cases}
$$

Letting $t \in \RR$, we can write

\begin{align*}
  \bm{x} \pars{t} &= \pars{t, \frac{1}{2} t + 1, - \frac{3}{2} t + 2} \\
  \bm{x} \pars{t} &= \boxed{\pars{0, 1, 2} + t \pars{0, \frac{1}{2}, - \frac{3}{2}}} \\
\end{align*}

This is of the form $\bm{x} \pars{t} = \bm{x}_0 + t \bm{v}$.
To find the distance, we must first normalize $\bm{v}$.

\begin{align*}
  \bm{u} &= \frac{\bm{v}}{\norm{\bm{v}}} \\
  &= \frac{\sqrt{10}}{5} \pars{0, \frac{1}{2}, - \frac{3}{2}} \\
\end{align*}

Suppose the point on the line closest to $\bm{a}_1$ is $\bm{p}$.
The shortest distance is then

\begin{align*}
  \norm{\bm{p} - \bm{a}_1}^2 &= \norm{\bm{x}_0 - \bm{a}_1}^2 - \norm{\pars{\bm{x}_0 - \bm{a}_1} \cdot \bm{u}}^2 \\
                             &= \norm{\pars{0, 1, 2} - \pars{1, 2, 1}}^2 - \frac{10}{25} \abs{\pars{\pars{0, 1, 2} - \pars{1, 2, 1}} \cdot \pars{0, \frac{1}{2}, - \frac{3}{2}}}^2 \\
                             &= \norm{\pars{-1, -1, 1}}^2 - \frac{2}{5} \abs{\pars{-1, -1, 1} \cdot \pars{0, \frac{1}{2}, - \frac{3}{2}}} \\
                             &= 3 - \frac{2}{5} \pars{2} \\
                             &= \frac{13}{5} \\
  \norm{\bm{p} - \bm{a}_1} &= \boxed{\sqrt{\frac{13}{5}}} \\
\end{align*}

\subsection*{c}

\begin{align*}
  \bm{x} &= \bm{b}_1 + t \bm{a} \\
\end{align*}

Suppose the line is at $\bm{b}_1$ when $t=0$.
In addition, at $t=1$, suppose the line is at $\bm{b}_2$.
Then, $\bm{a} = \bm{b}_2 - \bm{b}_1 = \pars{0, -1, 1}$.
Therefore,

$$
\bm{x} = \pars{1, 1, 0} + t \pars{0, -1, 1}
$$

The vector equation of the line is therefore

\begin{align*}
  \bm{a} \times \pars{\bm{x} - \bm{b}_1} &= 0 \\
  \pars{0, -1, 1} \times \pars{\bm{x} - \pars{1, 1, 0}} &= 0 \\
  \pars{0, -1, 1} \times \pars{x - 1, y - 1, z} &= 0 \\
  \pars{-y-z+1, x-1, x-1} &=0 \\
\end{align*}

Clearly, $x=1$ from the $\bm{e}_2$ and $\bm{e}_3$ components of this equation.
In addition, $y + z = 1$ from the $\bm{e}_1$ component.

The equation for $P_1$ is $2x-y+z=1$.
Substituting $x=1$, we have

$$
\begin{cases}
  y + z = 1 \\
  y - z = 1 \\
\end{cases}
$$

Solving, we find $y=1$ and $z=0$.
Finally, the point of intersection is $\boxed{\pars{1, 1, 0}}$.

\problem{1.15}
Let the orthonormal basis be composed of vectors $\bm{a}, \bm{b}, \bm{c}$.
We let
$$\bm{c} := \frac{\bm{v}}{\norm{\bm{v}}} = \frac{1}{\sqrt{26}} \pars{1, 4, 3}$$

In addition, we define
$$
\bm{w} := \pars{-4, 1, 0}
$$

Note that $\bm{w}$ and $\bm{c}$ are orthogonal.
We normalize and let the resultant unit vector be $\bm{a}$.

$$
\bm{a} := \frac{1}{\sqrt{17}} \pars{-4, 1, 0}
$$

To make $\bm{b}$ orthogonal to the other two vectors, we can compute the final vector:

\begin{align*}
  \bm{y} & := \bm{v} \times \bm{w} \\
         &= \pars{-3, -12, 15} \\
  \bm{b} &= \frac{\bm{y}}{\norm{\bm{y}}} \\
  \bm{b} &= \frac{1}{\sqrt{378}} \pars{-3, -12, 15} \\
\end{align*}


\problem{1.16}

Let the orthonormal basis be $\bm{u}_1, \bm{u}_2, \bm{u}_3$.

\begin{align*}
  \bm{u}_1 &= \frac{\bm{a}}{\norm{\bm{a}}} \\
           &= \frac{1}{\sqrt{26}} \pars{1, 4, 3} \\
  \bm{u}_2 &= \frac{\bm{b}}{\norm{\bm{b}}} \\
           &= \frac{1}{\sqrt{14}} \pars{3, 2, 1} \\
  \bm{c} &= \bm{a} \times \bm{b} \\
           &= \pars{-2, 8, -10} \\
  \bm{u}_3 &= \frac{\bm{c}}{\norm{\bm{c}}} \\
  &= \frac{1}{\sqrt{168}} \pars{-2, 8, -10} \\
\end{align*}

\problem{1.17}

\subsection*{a}

Let $\pars{s, t} = \pars{0, 0}, \pars{0, 1}, \pars{1, 0}$ correspond to $\bm{p}_1, \bm{p}_2, \bm{p}_3$, respectively.
We see that $\bm{x}_0 = \bm{p}_1 = \pars{-2, 0, 2}$.
Further, we also see that $$\bm{v}_1 = \bm{p}_2 - \bm{p}_1 = \pars{3, -2, 0}$$ and $$\bm{v}_2 = \bm{p}_3 - \bm{p}_1 = \pars{5, -1, -4}$$

$$
\bm{x} \pars{s, t} = \pars{-2, 0, 2} + s \pars{3, -2, 0} + t \pars{5, -1, -4}
$$

\subsection*{b}

Let $u=0$ at $\bm{x}_0$, so $\bm{z}_0 = \bm{x}_0 = \boxed{\pars{1, 4, -2}}$.
Letting $u=1$ at $\bm{x}_1$ lets us see that $\bm{w} = \bm{z}_1 - \bm{z}_0 = \boxed{\pars{-1, -7, 3}}$.

\subsection*{c}
We can compute the normal vector by

\begin{align*}
  \bm{n} &= \bm{v}_1 \times \bm{v}_2 \\
         &= \pars{3, -2, 0} \times \pars{5, -1, -4} \\
         &= \pars{8, 12, 7} \\
\end{align*}

We know that $\bm{n} \cdot \bm{x} + d = 0$ is the general vector equation for the plane.
Substituting $\bm{x} = \bm{p}_1$, we see that $d=2$.
We can then expand,
$$
\boxed{8x + 12y + 7z + 2 =0}
$$

\subsection*{d}

The general vector equation for a line is
\begin{align*}
  \bm{w} \times \pars{\bm{z} - \bm{z}_0} &= 0 \\
  \pars{-1, -7, 3} \times \pars{\pars{x, y, z} - \pars{1, 4, -2}} &= 0 \\
  \pars{-1, -7, 3} \times \pars{x - 1, y - 4, z + 2} &= 0 \\
  \pars{-2 - 3y - 7z, -1+3x+z, -3 + 7x - y} &= 0 \\
\end{align*}

We therefore have the system,

$$
\begin{cases}
  3y + 7z = -2 \\
  3x + z = 1 \\
  7x - y = 3 \\
\end{cases}
$$

\subsection*{e}
Solving for $y$ in the last equation,

$$
y = 7x - 3
$$

Substituting into the equation for the plane,

$$
\begin{cases}
  92x + 7z + 2 = 0 \\
  3x + z = 1 \\
\end{cases}
$$

Solving and resubstituting, we find $\boxed{\pars{\frac{24}{71}, - \frac{24}{71}, - \frac{10}{71}}}$.

\subsection*{f}

We must first normalize $\bm{w}$.

\begin{align*}
  \bm{u} &= \frac{\bm{w}}{\norm{\bm{w}}} \\
  &= \frac{1}{\sqrt{59}} \pars{-1, -7, 3} \\
\end{align*}

Suppose the point on the line closest to $\bm{p}_1$ is $\bm{q}$.

\begin{align*}
  \norm{\bm{p}_1 - \bm{q}} &= \norm{\pars{\bm{x}_0 - \bm{p}_1} \times \bm{u}} \\
                           &= \frac{1}{\sqrt{59}} \norm{\pars{\pars{1, 4, -2} - \pars{-1, -3, 0}} \times \pars{-1, -7, 3}} \\
                           &= \frac{1}{\sqrt{59}} \norm{\pars{2, 7, -2} \times \pars{-1, -7, 3}} \\
                           &= \frac{1}{\sqrt{59}} \norm{\pars{7, -4, -7}} \\
                           &= \boxed{\sqrt{\frac{114}{59}}} \\
\end{align*}


\subsection*{g}

We must first normalize $\bm{n}$.

\begin{align*}
  \bm{u} &= \frac{\bm{n}}{\norm{\bm{n}}} \\
  &= \frac{1}{\sqrt{257}} \pars{8, 12, 7} \\
\end{align*}

The distance is then

\begin{align*}
  \abs{\pars{\bm{x}_0 - \bm{z}_0} \cdot \bm{u}} &= \frac{1}{\sqrt{257}} \abs{\pars{-2, 0, 2} - \pars{1, 4, -2} \cdot \pars{8, 12, 7}} \\
                                                &= \frac{1}{\sqrt{257}} \abs{\pars{-3, -4, 4} \cdot \pars{8, 12, 7}} \\
                                                &= \frac{1}{\sqrt{257}} \abs{-44} \\
                                                &= \boxed{\frac{44}{\sqrt{257}}} \\
\end{align*}

\end{document}