\documentclass{article}

\usepackage[stdmargin, noindent]{../../rajeev}
\usepackage{mathtools}
\usepackage{dirtytalk}

\pagestyle{fancy}
\rhead{\today}
\lhead{Math 291H HW \#2}

\begin{document}

\begin{center}
    \Large \textbf{Math 291H Homework \#2}
\end{center}
\begin{center}
    \Large Rajeev Atla
\end{center}


Honors Pledge Statement: \say{The writeup of this submission is my own work alone.}

\problem{1.19}

\subsection*{a}

\begin{align*}
  \pars{1, 1, 0} + t \pars{1, -1, 2} &= \pars{2, 0, 2} + s \pars{-1, 1, 0} \\
  \pars{t, -t, 2t} - \pars{-s, s, 0} &= \pars{1, -1, 2} \\
  \pars{t + s, -t-s, 2t} &= \pars{1, -1, 2} \\
\end{align*}

This implies that $t=1$, which means that $s=0$.
Resubstituting, we find the point of intersection to be \boxed{\pars{2, 0, 2}}.

\subsection*{b}

Let $\bm{v}_1 = \pars{1, -1, 2}$ and $\bm{v}_2 = \pars{-1, 1, 0}$.
Both these vectors are in the plane.
We next find the normal vector $\bm{n}$.

$$
\bm{n} = \bm{v}_1 \times \bm{v}_2 = \pars{-2, -2, 0}
$$

The general form of a plane is $\bm{n} \cdot \bm{x} = d$.
Substituting $\bm{x} = \pars{1, 1, 0}$, we see that $d=-4$.
The equation for the plane is therefore

$$
-2 x + -2y = -4
$$

We can simplify this a little to get

$$
\boxed{x + y = 2}
$$

\problem{1.20}
Defining $\bm{n} := \pars{2, -1, 3}$ and $\bm{x}_0 := \pars{2, 0, 0}$, we can write the given equation as

$$
\bm{n} \cdot \pars{\bm{x} - \bm{x}_0} = 0
$$

We normalize $\bm{n}$, getting

$$
\bm{u} = \frac{1}{\sqrt{14}} \pars{2, -1, 3}
$$

The minimal distance between this plane and $\bm{p}$ is then

$$
\abs{\pars{\bm{x}_0 - \bm{p}} \cdot \bm{u}} = \frac{1}{\sqrt{14}} \abs{\pars{2, 3, 0} \cdot \pars{2, -1, 3}} = \boxed{\frac{1}{\sqrt{14}}}
$$


\problem{1.24}

Let $k \in \RR$, so that

$$
h_{\bm{u}} \pars{\bm{x}} = k \bm{e}_1
$$

Since $\norm{h_{\bm{u}} \pars{\bm{x}}} = \norm{\bm{x}}$, \boxed{k= \pm 7}.

By construction of the Householder reflection,

\begin{align*}
  \bm{u} &= \pm \frac{\bm{x} - k \bm{e}_1 }{\norm{\bm{x} - k \bm{e}_1}} \\
         &= \pm \frac{\pars{5 \pm 7, 2, 4, 2}}{\norm{\pars{5 \pm 7, 2, 4, 2}}} \\
  &= \boxed{\frac{\pars{6, 1, 2, 1}}{\sqrt{42}}, -\frac{\pars{6, 1, 2, 1}}{\sqrt{42}}, \frac{\pars{-1, 1, 2, 1}}{\sqrt{10}}, - \frac{\pars{-1, 1, 2, 1}}{\sqrt{10}} }\\
\end{align*}

(In the first two lines, the outer $\pm$ signs are independent of the inner ones.)

\problem{1.27}
We define the orthonormal basis

\begin{align*}
  \bm{u}_1 &= \frac{\bm{v}_1 \times \bm{v}_2}{\norm{\bm{v}_1 \times \bm{v}_2}} \\
           &= \frac{1}{\sqrt{14}}\pars{-3, -2, 1} \\
  \bm{u}_2 &= \frac{1}{\norm{\bm{v}_1}} \bm{v}_1 \times \bm{u}_1 \\
           &= \frac{1}{ \sqrt{10}} \pars{-1, 0, -3} \\
  \bm{u}_3 &= \bm{u}_1 \times \bm{u}_2 \\
           &= \frac{1}{\sqrt{35}} \pars{3, -5, -1} \\
\end{align*}

We can then define

$$
\bm{b} = \bm{x}_1 - \bm{x}_2 = \pars{0, 3, 1}
$$

We then have

$$
t = \frac{\bm{b} \cdot \bm{u}_2}{\bm{v}_2 \cdot \bm{u}_2} = \frac{-5}{10} = - \frac{1}{2}
$$

The point on the first line that corresponds to this $t$-value is
$$
\boxed{\pars{\frac{3}{2}, -\frac{1}{2}, -\frac{5}{2}}}
$$

Similarly,

$$
s = \frac{ t \pars{\bm{v}_2 - \bm{b} } \cdot \bm{u}_3 }{\bm{v}_1 \cdot \bm{u}_3} = \frac{- \frac{1}{2} \pars{-1, 0, 2} \cdot \pars{3, -5, -1}}{\pars{1, -4, -2} \cdot \pars{3, -5, -1}} = \frac{\frac{5}{2}}{25}  = \frac{1}{10}
$$

$$
\boxed{\pars{\frac{33}{10}, \frac{3}{2}, \frac{9}{10}}}
$$

The distance between these points is

$$
\abs{\bm{b} \cdot \bm{u}_1} = \boxed{\frac{5}{\sqrt{14}}}
$$


\problem{1.29}

\subsection*{a}

It's easy to see that $\norm{\bm{u}_i} = 1$.
In addition, we can compute

\begin{align*}
  \bm{u}_1 \cdot \bm{u}_2 &= \frac{1}{9} \pars{1 \cdot 2 + 2 \cdot 1 - 2 \cdot 2} = 0 \\
  \bm{u}_1 \cdot \bm{u}_3 &= \frac{1}{9} \pars{1 \cdot 2 + 2 \cdot \pars{-2} -2 \cdot \pars{-1}} = 0 \\
  \bm{u}_2 \cdot \bm{u}_3 &= \frac{1}{9} \pars{2 \cdot 2 + 1 \cdot \pars{-2} + 2 \cdot \pars{-1}} = 0 \\
\end{align*}

We see that

$$
\bm{u}_1 \times \bm{u}_2 = \frac{1}{3} \pars{2, -2, -1} = \bm{u}_3
$$

Therefore, this is a right-handed orthonormal basis.

\subsection*{b}

$$
\bm{u} = \frac{\bm{u}_1 - \bm{e}_1}{\norm{\bm{u}_1 - \bm{e}_1}} = \frac{\frac{1}{3} \pars{-2, 2, -2}}{\frac{1}{3} \norm{\pars{-2, 2, -2}}} = \boxed{\frac{1}{\sqrt{3}} \pars{-1, 1, -1}}
$$

\subsection*{c}

\begin{align*}
  h_{\bm{u}} \pars{\bm{u}_2} &= \bm{u}_2 - 2 \pars{\bm{u}_2 \cdot \bm{u}} \bm{u} \\
                             &= \frac{1}{3} \pars{2, 1, 2} - \frac{2}{9} \pars{-3} \pars{-1, 1, -1} \\
                             &= \pars{\frac{2}{3}, \frac{2}{3}, \frac{1}{3}} + \pars{-\frac{2}{3}, \frac{2}{3}, -\frac{2}{3}} \\
                             &= \boxed{\pars{0, 1, 0} = \bm{e}_2} \\
  h_{\bm{u}} \pars{\bm{u}_3} &= \bm{u}_3 - 2 \pars{\bm{u}_3 \cdot \bm{u}} \bm{u} \\
                             &= \frac{1}{3} \pars{2, -2, -1} - \frac{2}{9} \pars{-3} \pars{-1, 1, -1} \\
                             &= \pars{\frac{2}{3}, - \frac{2}{3}, - \frac{1}{3}} + \pars{- \frac{2}{3}, \frac{2}{3}, - \frac{2}{3}} \\
                             &= \boxed{\pars{0, 0, 1} = \bm{e}_1} \\
\end{align*}



\problem{1.32}

\subsection*{a}

Define $V := V_1 \cap V_2$.
Suppose we have vectors $\bm{r}_1, \bm{r}_2 \in V$ and scalars $a, b \in \RR$.
By definition, $\bm{r}_1 \in V_1, V_2$ and $\bm{r}_2 \in V_1, V_2$.
By definition of a subspace, $a \bm{r}_1 \in V_1, V_2$ and $b \bm{r}_1 \in V_1, V_2$.
Since these vectors are in both subspaces, we can say that

$$
a \bm{r}_1 + b \bm{r}_2 \in V_1, V_2
$$

This is what we wanted to show.

\subsection*{b}

If we scale up $\bm{z}$ by a factor of $\alpha$, we get

$$
\alpha \bm{z} = \alpha \bm{x} + \alpha \bm{y}
$$

Since $\alpha \bm{x} \in V_1$ and $\alpha \bm{x} \in V_2$ by the definition of a subspace, $\alpha \bm{z} \in V_1 + V_2$ by the construction of $V_1 + V_2$.
Similarly, let

$$
\bm{z'} = \bm{z}_1 + \bm{z}_2 = \bm{x}_1 + \bm{y}_1 + \bm{x}_2 + \bm{y}_2 = \pars{\bm{x}_1 + \bm{x}_1} + \pars{\bm{y}_1 + \bm{y}_2}
$$

By the definition of a subspace, $\bm{x}_1 + \bm{x}_2 \in V_1$.
Similarly, $\bm{y}_1 + \bm{y}_2 \in V_2$.
Therefore, by construction, $\bm{z}_1 + \bm{z}_2 \in V_1 + V_2$.

\problem{1.33}

Define $V_3 := V_1 + V_2$ and $V := V_1 \cap V_2$.
We then define an orthonormal basis for $V$.

$$
\set{\bm{u}_1, \dots, \bm{u}_p}
$$

Therefore, $\dim \pars{V} = p$.
We can extend $\set{\bm{u}_1, \dots, \bm{u}_p}$ to find orthonormal bases for $V_1$ and $V_2$.
By definition, $V \subseteq V_1$ and $V \subseteq V_2$.
Therefore, $\spn V \subseteq \spn V_1$ and $\spn V \subseteq \spn V_2$.
Similarly, $\spn V_1 \subseteq \spn V_3$ and $\spn V_2 \subseteq V_3$.
We can therefore write and define

\begin{align*}
  S &:= \set{\bm{u}_1, \cdots, \bm{u}_p, \bm{v}_1, \cdots, \bm{v}_q} \\
  T &:= \set{\bm{u}_1, \cdots, \bm{u}_p, \bm{w}_1, \cdots, \bm{w}_r} \\
  V_1 &= \spn \set{\bm{u}_1, \cdots, \bm{u}_p, \bm{v}_1, \cdots, \bm{v}_q} \\
  V_2 &= \spn \set{\bm{u}_1, \cdots, \bm{u}_p, \bm{w}_1, \cdots, \bm{w}_r} \\
\end{align*}

It's easy to see that $V_1 \subseteq \spn \pars{S \cup T}$ and $V_2 \subseteq \spn \pars{S \cup T}$.
Therefore, $V_3 \subseteq \spn \pars{S \cup T} $.
In addition, $\pars{S \cup T} \subseteq V_3$, so $\spn \pars{S \cup T} \subseteq V_3$.
Therefore,

\begin{align*}
  V_3 &= \spn \pars{S \cup T} \\
      &= \spn \set{\bm{u}_1, \cdots, \bm{u}_p, \bm{v}_1, \cdots, \bm{v}_q, \bm{w}_1, \cdots, \bm{w}_r} \\
  \dim V_3 &= p+q+r \\
\end{align*}

We can now verify that

\begin{align*}
  \dim V_3 + \dim V &= \dim V_1 + \dim V_2 \\
  \pars{p + q + r} + \pars{p} &= \pars{p+q} + \pars{p+r} \\
\end{align*}










\end{document}