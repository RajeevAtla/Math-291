\documentclass{article}

\usepackage[stdmargin, noindent]{../../rajeev}
\usepackage{mathtools}
\usepackage{dirtytalk}

\pagestyle{fancy}
\rhead{\today}
\lhead{Math 291H HW \#3}

\begin{document}

\begin{center}
    \Large \textbf{Math 291H Homework \#3}
  \end{center}
\begin{center}
    \Large Rajeev Atla
  \end{center}


  
Honors Pledge Statement: \say{The writeup of this submission is my own work alone.}

\problem{2.1}

\subsection*{a}

\begin{align*}
  \bm{x}' \pars{t} &= \pars{x' \pars{t}, y' \pars{t}} \\
                   &= \boxed{\pars{1, 2t}} \\
  \bm{x}'' \pars{t} &= \boxed{\pars{0, 2}} \\
\end{align*}

\subsection*{b}

\begin{align*}
  v \pars{t} &= \norm{\bm{v} \pars{t}} \\
             &= \boxed{\sqrt{1 + 4t^2}} \\
  \bm{T} \pars{t} &= \frac{\bm{v} \pars{t}}{v \pars{t}} \\
  &= \boxed{\frac{1}{\sqrt{1 + 4t^2}} \pars{1, 2t} } \\
\end{align*}


\subsection*{c}
At $t=1$, $\bm{x} = \pars{2, 1}$.

\begin{align*}
  \bm{y} \pars{t} & \approx \pars{2, 1} + \pars{t-1} \pars{1, 2} \\
  &= \boxed{\pars{1, -1} + t \pars{1, 2}} \\
\end{align*}


\problem{2.3}
We need to prove two preliminary results in order to solve this problem: both the product and sum of two continuous functions are continuous.

\subsection*{Sums}

If $\epsilon > 0$, then we must necessarily have $\frac{\epsilon}{2} > 0$.
If $\bm{x} \pars{t}$ and $\bm{y} \pars{t}$ are both continuous over $\RR$,

\begin{align*}
  \forall \epsilon > 0, \exists \delta_{\bm{x}} > 0 &: \abs{t - t_0} < \delta_{\bm{x}} \implies \norm{ \bm{x} \pars{t} - \bm{x} \pars{t_0}} < \frac{\epsilon}{2} \\
  \forall \epsilon > 0, \exists \delta_{\bm{y}} > 0 &: \abs{t - t_0} < \delta_{\bm{y}} \implies \norm{\bm{y} \pars{t} - \bm{y} \pars{t_0}} < \frac{\epsilon}{2} \\
\end{align*}

Define $\bm{z} \pars{t} := \bm{x} \pars{t} + \bm{y} \pars{t}$.
Using the triangle inequality,

\begin{align*}
  \norm{\bm{z} \pars{t} - \bm{z} \pars{t_0}} & = \norm{\bm{x} \pars{t} + \bm{y} \pars{t} - \bm{x} \pars{t_0} - \bm{y} \pars{t_0}} \\
                                             & = \norm{\bm{x} \pars{t} - \bm{x} \pars{t_0} + \bm{y} \pars{t} - \bm{y} \pars{t_0}} \\
                                             & \leq \norm{\bm{x} \pars{t} - \bm{x} \pars{t_0}} + \norm{\bm{y} \pars{t} - \bm{y} \pars{t_0}} \\
                                            & < 2 \pars{\frac{\epsilon}{2}} \\
                                            & = \epsilon \\
\end{align*}

Define $\delta_{\bm{z}} := \min \set{\delta_{\bm{x}}, \delta_{\bm{y}}}$.
We have

$$
\forall \epsilon > 0, \exists \delta_{\bm{z}} > 0 : \abs{t - t_0} < \delta_{\bm{z}} \implies \norm{ \bm{z} \pars{t} - \bm{z} \pars{t_0}} < \epsilon
$$
This shows that $\bm{z} \pars{t}$ is continuous.

\subsection*{Products}

Let $x \pars{t}, y \pars{t}$ be real-valued continuous functions over $\RR$.
Then,

\begin{align*}
    \forall \epsilon_{x} > 0, \exists \delta_{x} > 0 &: \abs{t - t_0} < \delta_{x} \implies \abs{ x \pars{t} - x \pars{t_0}} < \epsilon_x \\
  \forall \epsilon_{y} > 0, \exists \delta_{y} > 0 &: \abs{t - t_0} < \delta_{y} \implies \abs{y \pars{t} - y \pars{t_0}} < \epsilon_y \\
\end{align*}

Define $w \pars{t} := x \pars{t} y \pars{t}$.
We wish to show that

$$
\forall \epsilon_{w} > 0, \exists \delta_{w} > 0 : \abs{t - t_0} < \delta_{w} \implies \abs{ w \pars{t} - w \pars{t_0}} < \epsilon_w
$$

We have

\begin{align*}
  \abs{w \pars{t} - w \pars{t_0}} & = \abs{x \pars{t} y \pars{t} - x \pars{t_0} y \pars{t_0}} \\
                                  & = \abs{x \pars{t} y \pars{t} - x \pars{t} y \pars{t_0} + x \pars{t} y \pars{t_0} - x \pars{t_0} y \pars{t_0}} \\
                                  & \leq \abs{x \pars{t} y \pars{t} - x \pars{t} y \pars{t_0}} + \abs{x \pars{t} y \pars{t_0} - x \pars{t_0} y \pars{t_0}} \\
                                  & = \abs{x \pars{t}} \abs{y \pars{t} - y \pars{t_0}} + \abs{y \pars{t_0}} \abs{x \pars{t} - x \pars{t_0}} \\
                                  & < \epsilon_y \abs{x \pars{t}} + \epsilon_x \abs{y \pars{t_0}} \\
\end{align*}

Here, we have used the triangle inequality and the above definition of the continuity of $x$ and $y$.
Using the continuity of $x \pars{t}$, we can pick $\delta_x$, using $k \in \RR$ to avoid having a denominator of 0.

\begin{align*}
  \epsilon_x &= \frac{\epsilon_w}{2 \abs{y \pars{t_0}} + k},\ k > 0 \\
  \epsilon_x \abs{y \pars{t_0}} &= \frac{\epsilon_w \abs{y \pars{t_0}}}{2 \abs{y \pars{t_0}} + k} \\
  & < \frac{\epsilon_w}{2} \\
\end{align*}

We then note by the reverse triangle inequality,

$$
  \abs{x \pars{t} - x \pars{t_0}} \geq \abs{x \pars{t}} - \abs{x \pars{t_0}} \\
$$

In addition, using continuity, we can pick $\delta_x$ such that $\abs{x \pars{t} - x \pars{t_0}} < \epsilon_w$.
Combining with the above inequality, we see that

\begin{align*}
  \abs{x \pars{t}} - \abs{x \pars{t_0}} &< \epsilon_w \\
  \abs{x \pars{t}} &< \abs{x \pars{t_0}} + \epsilon_w \\
\end{align*}

We then pick $\delta_y$ such that

\begin{align*}
  \epsilon_y &= \frac{\epsilon_w}{2 \pars{\epsilon_w + \abs{x \pars{t_0}}}} \\
  \epsilon_y \abs{x \pars{t}} &= \frac{\epsilon_w \abs{x \pars{t}}}{2 \pars{\epsilon_w + \abs{x \pars{t_0}}}} \\
             & < \frac{\epsilon_w}{2} \pars{\frac{\epsilon_w + \abs{x \pars{t_0}}}{\epsilon_w + \abs{x \pars{t_0}}}} \\
             &= \frac{\epsilon_w}{2} \\
\end{align*}

We then have

\begin{align*}
  \epsilon_y \abs{x \pars{t}} + \epsilon_x \abs{y \pars{t_0}} & < 2 \pars{\frac{\epsilon_w}{2}} \\
  &= \epsilon_w
\end{align*}

Putting together all the inequalities, we have

$$
\abs{w \pars{t} - w \pars{t_0}} < \epsilon_w
$$

This establishes that $w \pars{t}$ is a continuous function in $\RR$.


\subsection*{Dot Product}
We show that the dot product is a combination of additive and multiplicative operations on the entry functions of $\bm{x} \pars{t}$ and $\bm{y} \pars{t}$.
Recall that in order for $\bm{x} \pars{t}$, each of $\set{x_i \pars{t}}$ must also be a continuous function.
The same goes for $\bm{y} \pars{t}$.

Define $v \pars{t} := \bm{x} \pars{t} \cdot \bm{y} \pars{t}$.
By definition

\begin{align*}
  v \pars{t} &= \bm{x} \pars{t} \cdot \bm{y} \pars{t} \\
             &= \sum \limits_{i=1}^{i=n} x_i \pars{t} y_i \pars{t} \\
\end{align*}

Further, define

$$
v_{k} \pars{t} := \sum \limits_{i=1}^{k \leq n} x_i \pars{t} y_i \pars{t}
$$

We see that $v_1 \pars{t} = x_1 \pars{t} y_1 \pars{t}$, the product of two continuous functions, making it a continuous function.
Similarly,
\begin{align*}
  v_2 \pars{t} &= x_1 \pars{t} y_1 \pars{t} + x_2 \pars{t} y_2 \pars{t} \\
               &= v_1 \pars{t} + x_2 \pars{t} y_2 \pars{t} \\
\end{align*}

Since both $x_2 \pars{t}$ and $y_2 \pars{t}$ are both continuous functions, their product is also a continuous function.
In addition, we have already shown that $v_1 \pars{t}$ is a continuous function.
Therefore, $v_2 \pars{t}$ is a continuos function.
In general for $k \leq n$,

$$
v_k \pars{t} = v_{k-1} \pars{t} + x_k \pars{t} y_k \pars{t}
$$

Each term is a continouus function, so $v_k \pars{t}$ is also a continuous function.
In the $k=n$ case, this becomes the dot product, completing the proof.



\subsection*{Cross Product}
In $\RR^3$, the cross product is

\begin{align*}
  \bm{x} \pars{t} \times \bm{y} \pars{t} &=
                                           \begin{vmatrix}
                                             \bm{e}_1 & \bm{e}_2 & \bm{e}_3 \\
                                             x_1 \pars{t} & x_2 \pars{t} & x_3 \pars{t} \\
                                             y_1 \pars{t} & y_2 \pars{t} & y_3 \pars{t} \\
                                           \end{vmatrix} \\
                                         &= \pars{x_2 \pars{t} y_3 \pars{t} - x_3 \pars{t} y_2 \pars{t}} \bm{e}_1 + \pars{x_3 \pars{t} y_1 \pars{t} - x_1 \pars{t} y_3 \pars{t}} \bm{e}_2 + \pars{x_1 \pars{t} y_2 \pars{t} - x_2 \pars{t} y_1 \pars{t}} \bm{e}_3 \\
\end{align*}

A vector function is continuous if and only if each of its entry functions is also continuous.
Therefore, it suffices to show that each entry function is continuous.
Each entry function is a combination of products of continuous functions and differences between these products.
Therefore, the cross product is continuous.

\problem{2.4}

\subsection*{a}

\begin{align*}
  \bm{v} \pars{t} &= \boxed{\pars{- \sin \pars{t}, \cos \pars{t}, \frac{1}{r}}} \\
  \bm{a} \pars{t} &= \boxed{\pars{- \cos \pars{t}, - \sin \pars{t}, 0}} \\
\end{align*}

\subsection*{b}

\begin{align*}
  v \pars{t} &= \sqrt{\sin^2 t + \cos^2 t + \frac{1}{r^2}} \\
             &= \boxed{\sqrt{1 + \frac{1}{r^2}}} \\
  \bm{T} \pars{t} &= \frac{\bm{v} \pars{t}}{v \pars{t}} \\
             &= \boxed{ \frac{r}{\sqrt{1+r^2}} \pars{- \sin \pars{t}, \cos \pars{t}, \frac{1}{r}}} \\
\end{align*}

\subsection*{c}

\begin{align*}
  \bm{a}_{\parallel} \pars{t} &= v' \pars{t} \bm{T} \pars{t} \\
  &= 0 \\
  \bm{a}_{\perp} \pars{t} &= v \pars{t} \bm{T}' \pars{t} \\
                              &= \pars{\sqrt{1 + \frac{1}{r^2}}} \pars{\frac{r}{\sqrt{1+r^2}} \pars{- \cos \pars{t}, - \sin \pars{t}, 0}} \\
                              &= \pars{- \cos \pars{t}, - \sin \pars{t}, 0} \\
  \bm{N} \pars{t} &= \frac{\bm{a}_{\perp} \pars{t}}{\norm{\bm{a}_{\perp} \pars{t}}} \\
                              &= \boxed{\pars{- \cos \pars{t}, - \sin \pars{t}, 0}} \\
  \kappa \pars{t} &= \frac{ \norm{\bm{a}_{\perp} \pars{t}} }{v^2 \pars{t}} \\
                              &= \frac{1}{1 + \frac{1}{r^2}} \\
                              &= \boxed{\frac{r^2}{1+r^2}} \\
  \bm{B} \pars{t} &= \bm{T} \pars{t} \times \bm{N} \pars{t} \\
                              &= \boxed{\frac{1}{\sqrt{1+r^2}} \pars{\sin \pars{t}, - \cos \pars{t}, r}} \\
  \tau \pars{t} &= \frac{1}{v^6 \pars{t} \kappa^2 \pars{t}} \pars{\bm{x}' \pars{t} \times \bm{x}'' \pars{t}} \cdot \bm{x}''' \pars{t} \\
                              &= \frac{1}{\pars{1 + \frac{1}{r^2}}^3} \pars{1 + \frac{1}{r^2}}^2 \pars{\pars{- \sin \pars{t}, \cos \pars{t}, \frac{1}{r}} \times \pars{- \cos \pars{t}, - \sin \pars{t}, 0}} \cdot \pars{\sin \pars{t}, - \cos \pars{t}, 0} \\
                              &= \frac{1}{1 + \frac{1}{r^2}} \pars{\frac{1}{r} \sin \pars{t}, - \frac{1}{r} \cos \pars{t}, 1} \cdot \pars{\sin \pars{t}, - \cos \pars{t}, 0} \\
                              &= \frac{r^2}{1+r^2} \pars{\frac{1}{r} \sin^2 \pars{t} + \frac{1}{r} \cos^2 \pars{t}} \\
                              &= \boxed{\frac{r}{1+r^2}} \\
\end{align*}

\subsection*{d}

\begin{align*}
  \boldsymbol{\omega} \pars{t} &= \tau \pars{t} \bm{T} \pars{t} + \kappa \pars{t} \bm{B} \pars{t} \\
                               &= \pars{\frac{r}{1+r^2}} \pars{\frac{r}{\sqrt{1 + r^2}} \pars{- \sin \pars{t}, \cos \pars{t}, \frac{1}{r}}} + \pars{\frac{r^2}{1+r^2}} \pars{\frac{1}{\sqrt{1+r^2}} \pars{\sin \pars{t}, - \cos \pars{t}, r}} \\
                               &= \frac{r^2}{\pars{1+r^2}^{\frac{3}{2}}} \pars{0, 0, r + \frac{1}{r}} \\
  &= \boxed{\frac{r}{\sqrt{1+r^2}} \pars{0, 0, 1}} \\
\end{align*}


\subsection*{e}

\subsubsection*{Tangent Line}

\begin{align*}
  \bm{x} \pars{t} & \approx \bm{x} \pars{\frac{\pi}{4}} + \pars{t - \frac{\pi}{4}} \bm{v} \pars{\frac{\pi}{4}}\\
                  &= \pars{\frac{1}{\sqrt{2}}, \frac{1}{\sqrt{2}}, \frac{\pi}{4r}} + \pars{t - \frac{\pi}{4}} \pars{- \frac{1}{\sqrt{2}}, \frac{1}{\sqrt{2}}, \frac{1}{r}} \\
                  &= \boxed{\pars{\frac{4+\pi}{4 \sqrt{2}}, \frac{4 - \pi}{4 \sqrt{2}} , 0} + t \pars{- \frac{1}{\sqrt{2}}, \frac{1}{\sqrt{2}}, \frac{1}{r}}} \\
\end{align*}

\subsubsection*{Osculating Plane}
\begin{align*}
  \bm{B} \pars{\frac{\pi}{4}} \cdot \pars{\bm{x} \pars{t} - \bm{x} \pars{\frac{\pi}{4}}} &= 0 \\
  \bm{B} \pars{\frac{\pi}{4}} \cdot \bm{x} \pars{t} &= \bm{B} \pars{\frac{\pi}{4}} \cdot \bm{x} \pars{\frac{\pi}{4}} \\
  \frac{1}{\sqrt{1+r^2}} \pars{\frac{\sqrt{2}}{2}, - \frac{\sqrt{2}}{2}, r} \cdot \bm{x} \pars{t} &= \frac{1}{\sqrt{1+r^2}} \pars{\frac{\sqrt{2}}{2}, - \frac{\sqrt{2}}{2}, r} \cdot \pars{\frac{\sqrt{2}}{2}, \frac{\sqrt{2}}{2}, \frac{\pi}{4r}} \\
  \pars{2\sqrt{2}, - 2\sqrt{2}, 4r} \cdot \bm{x} \pars{t} &= \pi \\
  \boxed{2 x\sqrt{2} - 2y\sqrt{2} + 4rz = \pi} \\
\end{align*}

\subsubsection*{Intersection}

\begin{align*}
  x \pars{t} &= \frac{4+\pi}{4 \sqrt{2}} - \frac{t}{\sqrt{2}} \\
  y \pars{t} &= \frac{4-\pi}{4 \sqrt{2}} + \frac{t}{\sqrt{2}} \\
  z \pars{t} &= \frac{t}{r} \\
  2 x\sqrt{2} - 2y\sqrt{2} + 4rz &= \pi \\
\end{align*}

We see that upon substitution, we are left with a contradiction, meaning that the line and plane never intersect.

\problem{2.5}

\subsubsection*{a}

\begin{align*}
  \bm{v} \pars{t} &= \pars{e^t \cos t - e^t \sin t, e^t \sin t + e^t \cos t, e^t} \\
  v \pars{t} &= e^t \sqrt{\pars{\cos t - \sin t}^2 + \pars{\cos t + \sin t}^2 + 1} \\
  &= e^t \sqrt{3} \\
  s \pars{t} &= \int \limits_{t=0}^{t} e^t \sqrt{3}\ dt \\
  &= \pars{e^t -1 } \sqrt{3} \\
\end{align*}

In order to write an arc-length parametrization, we invert $s \pars{t}$ to get $t \pars{s}$.

\begin{align*}
  s &= \pars{e^t - 1} \sqrt{3} \\
  \frac{s}{\sqrt{3}} &= e^t - 1 \\
  t \pars{s} &= \ln \pars{\frac{s}{\sqrt{3}} + 1} \\
\end{align*}

Finally, we find $\bm{x} \pars{t \pars{s}}$.

\begin{align*}
  \bm{x} \pars{t \pars{s}} &= \pars{\pars{\frac{s}{\sqrt{3}} + 1} \cos \pars{\pars{\frac{s}{\sqrt{3}} + 1}}, \pars{\frac{s}{\sqrt{3}} + 1} \sin \pars{\pars{\frac{s}{\sqrt{3}} + 1}}, \frac{s}{\sqrt{3}} + 1} \\
\end{align*}

\subsection*{b}
\begin{align*}
  \bm{T} \pars{t} &= \frac{\bm{v} \pars{t}}{v \pars{t}} \\
                  &= \frac{1}{\sqrt{3}} \pars{\cos t - \sin t, \cos t + \sin t, 1} \\
  \bm{T}' \pars{t} &= \frac{1}{\sqrt{3}} \pars{- \sin t - \cos t, - \sin t + \cos t, 0} \\
  \bm{a}_{\perp} \pars{t} &= v \pars{t} \bm{T}' \pars{t} \\
                  &= e^t \pars{- \sin t - \cos t, - \sin t + \cos t, 0} \\
  \norm{\bm{a}_{\perp} \pars{t}} &= e^t \sqrt{2} \\
  \kappa \pars{t} &= \frac{\norm{\bm{a}_{\perp} \pars{t}}}{v^2 \pars{t}} \\
                  &= \frac{e^t \sqrt{2}}{3 e^{2t}} \\
                  &= \boxed{e^{-t} \frac{\sqrt{2}}{3}} \\
\end{align*}

\begin{align*}
  \bm{N} \pars{t} &= \frac{\bm{a}_{\perp}}{\norm{\bm{a}_{\perp}}} \\
                  &= \frac{1}{\sqrt{2}} \pars{- \sin t - \cos t, - \sin t + \cos t, 0} \\
  \bm{B} \pars{t} &= \bm{T} \pars{t} \times \bm{N} \pars{t} \\
                  &= \frac{1}{\sqrt{6}} \pars{\sin t - \cos t, - \sin t - \cos t, 2} \\
  \bm{B}' \pars{t} &= \frac{1}{\sqrt{6}} \pars{\cos t + \sin t, - \cos t + \sin t, 0} \\
  \bm{B}' \pars{t} &= - v \pars{t} \tau \pars{t} \bm{N} \pars{t} \\
  \frac{1}{\sqrt{6}} \pars{\cos t + \sin t, - \cos t + \sin t, 0} &= - \tau \pars{t} \pars{e^t \sqrt{3}} \pars{\frac{1}{\sqrt{2}} \pars{- \sin t - \cos t, - \sin t + \cos t, 0}} \\
  \tau \pars{t} &= \boxed{\frac{e^{-t}}{3 \sqrt{2}}} \\
\end{align*}

\subsection*{c}

\begin{align*}
  \bm{x} \pars{0} &= \pars{1, 0, 1} \\
  \bm{B} \pars{0} &= \frac{1}{\sqrt{6}} \pars{-1, -1, 2} \\
  \bm{B} \cdot \bm{x} &= \bm{B} \cdot \bm{x} \pars{0} \\
  \pars{-1, -1, 2} \cdot \pars{x, y, z} &= \pars{-1, -1, 2} \cdot \pars{1, 0, 1} \\
  -x - y + 2z &= 1 \\
\end{align*}


\problem{2.7}

\subsection*{a}

\begin{align*}
  \bm{x} \pars{t} &= \pars{2t, t^2, \frac{t^3}{3}} \\
  \bm{v} \pars{t} &= \pars{2, 2t, t^2} \\
  v \pars{t} &= \sqrt{4 + 4t^2 + t^4} \\
                  &= t^2 + 2 \\
  v' \pars{t} &= 2t \\
  \bm{T} \pars{t} &= \pars{\frac{2}{t^2+2}, \frac{2t}{t^2+2}, \frac{t^2}{t^2+2}} \\
  \bm{T}' \pars{t} &= \pars{- \frac{4t}{\pars{t^2+2}^2}, \frac{2 \pars{2- t^2}}{\pars{t^2+2}^2}, \frac{4}{\pars{t^2+2}^2}} \\
  \bm{a}_{\parallel} \pars{t} &= \pars{\frac{4t}{t^2+2}, \frac{4t^2}{t^2+2}, \frac{2t^3}{t^2+2}} \\
  \bm{a}_{\perp} \pars{t} &= \pars{- \frac{4t}{t^2 + 2}, \frac{2 \pars{2-t^2}}{t^2 + 2}, \frac{4}{t^2 + 2}} \\
  \norm{\bm{a}_{\perp} \pars{t}} &= \frac{2 \sqrt{t^4 + 8}}{t^2 + 2} \\
  \bm{N} \pars{t} &= \pars{-\frac{2t}{\sqrt{t^4 + 8}}, \frac{2-t^2}{\sqrt{t^4 + 8}}, \frac{2}{\sqrt{t^4 + 8}}} \\
  \bm{B} \pars{t} &= \pars{\frac{t \pars{t^3 - 2t + 2}}{\pars{t^2 + 2} \sqrt{t^4 + 8}}, - \frac{2 \pars{t^3 + 2}}{\pars{t^2 + 2} \sqrt{t^4 + 8}}, \frac{2}{\sqrt{t^4 + 8}} } \\
  \bm{B} \pars{1} &= \pars{\frac{1}{9}, - \frac{2}{3}, \frac{2}{3}} \\
  \bm{x} \pars{1} &= \pars{1, \frac{1}{2}, \frac{1}{3}} \\
  \bm{B} \pars{1} \cdot \bm{x} \pars{t} &= \bm{B} \pars{1} \cdot \bm{x} \pars{1} \\
  \pars{\frac{1}{9}, - \frac{2}{3}, \frac{2}{3}} \cdot \pars{x, y, z} &= \frac{1}{6} \\
  \boxed{\frac{1}{9} x - \frac{2}{3} y + \frac{2}{3} z = \frac{1}{6}} \\
\end{align*}

\subsection*{b}
The minimal distance is

\begin{align*}
  \abs{\pars{\bm{x} \pars{1} - \pars{0, 0, 0}} \cdot \bm{B} \pars{1}} &= \boxed{\frac{1}{6}} \\
\end{align*}


\problem{2.9}

\subsection*{a}

We substitute the parametrized equation into the Cartesian equation to verify tht the parametrization is valid.

\begin{align*}
  \frac{x^2}{a^2} + \frac{y^2}{b^2} &= 1 \\
  \frac{a^2 \cos^2 t}{a^2} + \frac{b^2 \sin^2 t}{b^2} &= 1\\
  \cos^2 t + \sin^2 t &= 1 \\
\end{align*}

This trigonometric identity is obviously true regardless of the value of $t$, so the parametrization works.

\subsection*{b}

\begin{align*}
  \bm{x} \pars{t} &= \pars{a \cos t, b \sin t} \\
  \bm{v} \pars{t} &= \pars{- a \sin t, b \cos t} \\
  v \pars{t} &= \sqrt{a^2 \sin^2 t + b^2 \cos^2 t} \\
  \bm{T} \pars{t} &= \pars{- \frac{a \sin t}{\sqrt{a^2 \sin^2 t + b^2 \cos^2 t}}, \frac{b \cos t}{\sqrt{a^2 \sin^2 t + b^2 \cos^2 t}}} \\
  \bm{T}' \pars{t} &= \pars{-\frac{ab^2 \cos t}{\pars{a^2 \sin^2 t + b^2 \cos^2 t}^{\frac{3}{2}}}, -\frac{a^2 b \sin t}{\pars{a^2 \sin^2 t + b^2 \cos^2 t}^{\frac{3}{2}}}} \\
  \bm{a}_{\perp} \pars{t} &= \pars{-\frac{ab^2 \cos t}{a^2 \sin^2 t + b^2 \cos^2 t}, -\frac{a^2 b \sin t}{a^2 \sin^2 t + b^2 \cos^2 t}} \\
  \norm{\bm{a}_{\perp} \pars{t}} &= \frac{ab}{\sqrt{a^2 \sin^2 t + b^2 \cos^2 t}} \\
  \kappa \pars{t} &= \boxed{\frac{ab}{\pars{a^2 \sin^2 t + b^2 \cos^2 t}^{\frac{3}{2}}}} \\
  \kappa ' \pars{t} &= \frac{3ab \pars{b^2 - a^2} \sin t \cos t}{\pars{a^2 \sin^2 t + b^2 \cos^2 t}^{\frac{5}{2}}} \\
\end{align*}

Let $n \in \ZZ$.
We see that the critical points are
$$
\kappa \pars{\pi n} = \frac{a}{b^2}
$$
and

$$
\kappa \pars{\frac{\pi}{2} \pars{2n+1}} = \frac{b}{a^2}
$$

There are 3 main cases:
\begin{itemize}
\item When $a=b$, the ellipse degenerates into a circle and the curvature is constant throughout.
\item When $a>b$, the ellipse is oriented along the $x$-axis.
  Here, $\frac{a}{b^2} > \frac{b}{a^2}$.
  This means that $t=\pi n$ are the maxima and $t = \frac{\pi}{2} \pars{2n+1}$ are the minima.
  These $t$-coordinates correspond to the points $\pars{\pm a, 0}$ and $\pars{0, \pm b}$.
\item When $a < b$, the ellipse is oriented along the $y$-axis.
  Here, $\frac{a}{b^2} < \frac{b}{a^2}$.
  This means that $t=\pi n$ are the minima and $t = \frac{\pi}{2} \pars{2n+1}$ are the maxima.
  These $t$-coordinates correspond to the points $\pars{\pm a, 0}$ and $\pars{0, \pm b}$.
\end{itemize}



\problem{2.11}


\begin{align*}
  \bm{x} \pars{t} &= \pars{t+1, t^2} \\
  \bm{v} \pars{t} &= \pars{1, 2t} \\
  v \pars{t} &= \sqrt{1+4t^2} \\
  s &= \int \limits_{-1}^{0} \sqrt{1+4t^2}\ dt \\
  & \boxed{\approx 1.479} \\
\end{align*}


\end{document}

